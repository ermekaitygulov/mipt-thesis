\documentclass{mipt-thesis-bs}
% Следующие две строки нужны только для biblatex. Для inline-библиографии их следует убрать.
\usepackage{mipt-thesis-biblatex}
\addbibresource{example.bib}

\title{Reinforcement Learning, Rozum}
\author{Айтыгулов Э.}
\supervisor{Панов А.\,И.}
%\referee{Петров Д.\,Е.}       % требуется только для mipt-thesis-ms
\groupnum{М05-878б}
\faculty{Факультет проблем физики и энергетики}
\department{Кафедра космической физики}

\begin{document}

\frontmatter
\titlecontents

\mainmatter


\chapter{Введение}

Здесь идет текст. Вот так выглядит ссылка на библиографию \cite{langmuir26}. Аналогично добавляются еще главы, внутри них можно объявлять секции с помощью \verb|\section|.


\chapter{RL}
\section{Марковский процесс принятия решений}


\chapter{DQN}


\chapter{DDPG}


\chapter{APEX}


\backmatter

% \printbib
% Следующие строки необходимо раскомментировать, а предыдущую закомментировать, если используется inline-библиография.
%\begin{thebibliography}{99}
%    \bibitem{langmuir26}
%        H. Mott-Smith, I. Langmuir. ``The theory of collectors in gaseous discharges''. \emph{Phys. Rev.} \textbf{28} (1926)
%\end{thebibliography}

\chapter{Благодарности}

Благодарности идут тут.

\end{document}
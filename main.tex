\documentclass{mipt-thesis-bs}
% Следующие две строки нужны только для biblatex. Для inline-библиографии их следует убрать.
\usepackage{mipt-thesis-biblatex}
\addbibresource{example.bib}

\title{Обучение с подкреплением с применением экспертных демонстраций}
\author{Айтыгулов Э.}
\supervisor{Осипов Г.\,С.}
%\referee{Петров Д.\,Е.}       % требуется только для mipt-thesis-ms
\groupnum{М05-878б}
\faculty{Физтех-школа Прикладной Математики и Информатики}
\department{Кафедра системных исследований}

\begin{document}

\frontmatter
\titlecontents

\chapter{Введение}

Цель данной работы применение методов обучения с подкреплением в задаче управления робототехническим манипулятором в симуляторе \cite{}. В работе предлагается метод ускорения процесса обучения и уменьшения затрачиваемых ресурсов для моделей, работающих с изображениями в качестве входных данных, с помощью моделей использующих данные доступные в симулляторе. 

Данный подход может быть применен в задачах по переносу моделей из симуллятора в реальность, когда модель обученная в симулляторе с помощью техник рандомизации или домен адаптации переносится на реального робота.

В качестве базовых алгоритмов использовались асинхронный алгоритм APEX DQN \cite{} и улучшенная версия APEX DDPG  \cite{}. Для симулляции окружающей среды использовался симуллятор Coppelia-Sim: с помощью стороннего пакета PyREP было реализованно окружение поддерживающее различные комбинации пространств состояний (картинка/(картинка, положение манипулятора)/ (положение, манипулятора, положение цели)

\chapter{Обзор Работ}



\mainmatter


\chapter{Обучение с подкреплением}

Обучение с подкреплением - способ обучения модели с помощью ее взаимодействия в некоторой окружающей среде, в которой она совершает действия, наблюдает ее состояния и получает положительный или отрицательный отклики (подкрепление). Алгоритмы обучения с подкреплением нацеленны на увеличение суммарной награды  
\section{Марковский процесс принятия решений}
Для описания всей системы используется 
\section{Функции значимости}


\chapter{DQN}
\section{Q-обучение}
\section{Стабилизация целевой переменной и оптимизация архитектуры нейронной сети}
\section{Приоритезированный буфер}


\chapter{TD3}
\section{Градиент стратегии}
\section{Исполнитель-Критик}
\section{Сглаживание апроксимации Q-функции}


\chapter{APEX}
\section{Архитектура}
\section{Исследование среды}

\chapter{Среда RozumEnv}


\backmatter

% \printbib
% Следующие строки необходимо раскомментировать, а предыдущую закомментировать, если используется inline-библиография.
%\begin{thebibliography}{99}
%    \bibitem{langmuir26}
%        H. Mott-Smith, I. Langmuir. ``The theory of collectors in gaseous discharges''. \emph{Phys. Rev.} \textbf{28} (1926)
%\end{thebibliography}

\chapter{Благодарности}

Благодарности идут тут.

\end{document}